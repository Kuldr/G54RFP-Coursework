\documentclass[a4paper]{article}
\usepackage{lipsum}
\usepackage{url}
\usepackage{graphicx}
\usepackage{listings}
\usepackage{indentfirst}
\usepackage{enumerate}
\usepackage{multicol}
\lstset{language=Haskell}
\usepackage[margin=2cm]{geometry}
\graphicspath{ {images/} }
\renewcommand{\familydefault}{\sfdefault}

\title{COMP4075/G54RFP Coursework Part II}
\date{5\textsuperscript{th} December 2018}
\author{Benjamin Charlton --- psybc3 --- 4262648}

\usepackage{fancyhdr}

\pagestyle{fancy}
\fancyhf{}
\lhead{Benjamin Charlton | psybc3 | 4262648}
\rhead{G54RFP}
\cfoot{\thepage}

\begin{document}

\maketitle

\section{Task II.1 --- Dining Philosophers}
\begin{itemize}
    \item Explain Design and Implementation
    \item Sample output
    \item Discuss STM in relation to Resource Hierarchy Solution and arbitrator solution on wikipedia
\end{itemize}

\subsection{Solution Design}
The problem is to develop a system simulating a number of philosophers at a table all alternating between thinking and eating.
The issue is the limited amount of eating implements (sporks were used in this solution as to distinguish from forking threads) and the fact each philosopher needs 2 sporks to eat.
\par
The main idea behind the solution is the use of Software Transactional Memory (STM), where each spork is held in a piece of STM\@.
If a philosopher is hungry they will wait until they can grab both sporks at the same time and then begin eating.
After they have finished eating they return the sporks to the table (adding them back into the relevant STM) so others can get them if required.

\subsection{Sample Output}
Here is some sample output for my solution running with 7 philosophers around the table.
It shows the first 50 lines of output and the time of each line being printed.
\begin{center}
    \begin{multicols}{2}
        \lstinputlisting[firstline=0, lastline=50]{Suppliments/dpOutput.txt}
    \end{multicols}
\end{center}

% \subsection{Implementation}
% \subsubsection{Helper Functions}


% \lstinputlisting[language=Haskell, firstline=5, lastline=9]{reportCode.hs}


\end{document}
